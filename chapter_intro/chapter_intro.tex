\chapter{Introduction}
\label{chap:intro}

\begin{refsection}

\section{Black Holes}

In virtually all of the events in our solar system, the differences between the predictions of general relativity and newtonian gravity are detectible via careful observation but hardly flamboyant. The relativistic corrections to Newtonian gravity do not lead to a radically different environment. General relativity's most exotic predictions come in the strong field regions of spacetime known as black holes, which have no analog in newtonian gravity. Even though general relativity has passed all weak-field tests with flying colors, it is not a given that it is correct when extrapolated to the strong field situations hardly experienced in the universe.

Perhaps the most interesting strong gravitational fields in the universe are black holes. The no hair theorem states that all asymptotically flat black holes can be parameterized by a mass $M$, specific angular momentum $a$, and a charge $Q$ and fall in the Kerr-Newman family.
%Kerr-Newman Black holes have a horizon, a light ring
An essential question that this works to address is:  how well does the Kerr-Newman metric describe astrophysical black holes?

Throughout this thesis, we use geometric units with $G=c=1$

%\section{Gravitational Waves}

%This thesis hopes to contribute to our understanding of how specific parts of the black hole geometry influence specific parts of the black hole waveform.

%\section{Quasinormal Modes}

%\section{Testing General Relativity}

%\section{Overview}








\printbibliography[heading=subbibliography]
\end{refsection}